\documentclass[preprint,superscriptaddress,
nofootinbib,amsmath,amssymb,aps]{revtex4-1}

% Use the lineno option to display guide line numbers if required.

\usepackage{graphicx}% Include figure files
\usepackage{dcolumn}% Align table columns on decimal point
\usepackage{bm}% bold math
%\usepackage{colortbl}
\usepackage[table]{xcolor}
%\usepackage[mathlines]{lineno}% Enable numbering of text and display math
%\linenumbers\relax % Commence numbering lines

\usepackage[utf8]{inputenc}
\usepackage[T1]{fontenc}

% Our packages
\usepackage{amsmath}
% \graphicspath{{figures/}}
\usepackage{makecell}
\renewcommand{\cellalign}{tl}
\newcommand{\george}[1]{\textcolor{red}{#1}}
\newcommand{\alex}[1]{\textcolor{blue}{#1}}
\newcommand{\kalel}[1]{\textcolor{orange}{#1}}
\usepackage{listings} % For listing code
\usepackage{mathptmx}
\usepackage{etoolbox}


\begin{document}
\hspace{-0.5cm}{\large Supplementary information for: Framework for global stability analysis of dynamical systems}\\*
George Datseris, Kalel Luiz Rossi, Alexander Wagemakers\\*
%\correspondingauthor{George Datseris.\\E-mail:  g.datseris@exeter.ac.uk}

\nopagebreak
\section{Software package flowchart}

\begin{figure*}[!h]
\centering 
\includegraphics[width=17.8cm]{figures/attractorsjl_overview}
\caption{Schematic overview of global stability analysis in the software Attractors.jl (part of DynamicalSystems.jl)}
\label{fig:continuation_algorithm}
\end{figure*}


\section{Simulated systems}
In this section we state the dynamic rules (equations of motion), parameter values, and state space boxes, used for all systems in the main text.

\subsection*{Chaotic Lorenz84} This model due to Lorenz~\cite{Lorenz84} is an extremely simplified representation of atmospheric flow as a low dimensional dynamical system with equations
\begin{align*}
\dot x &= - y^2 - z^2 - ax + aF, \\
\dot y &= xy - y - bxz + G, \\
\dot z &= bxy + xz - z
\end{align*}
and parameters $F=6.846, a=0.25, b=4.0$ and $G$ ranging from 1.34 to 1.37. The state space box and tessellation was from -3 to +3 discretized into 600 points in each dimension.

\subsection*{Climate toy model} The high-dimensional toy model of global climate is due to Gelbrecht et al.~\cite{Gelbrecht2021}. 
\begin{align*}
    \dot{X}_n &= (X_{n+1} - X_{n-2})X_{n-1} - X_n + F \left(1 + \beta \frac{T - \bar{T}}{\Delta_T}\right) , \quad n = 1, \dots, N; \; n\pm N \equiv i \\ 
    \dot{T} &= S\left(1 - a_0 + \frac{a_1}{2}\tanh \left( T - \bar{T}\right)\right) - \sigma T^4 - \alpha\left( \frac{\mathcal{E} (\mathbf{X})}{0.6 F^{\tfrac{4}{3}}} - 1 \right) \\
    \mathcal{E}(X) &= \frac{1}{2N}\sum_{n=1}^{N}X_n^2
\end{align*}
with parameter values identical to those in Table 1 of Ref.~\cite{Gelbrecht2021}, however we used $N=32$ $X$ variables. The parameter we varied was the solar constant $S$ from 5 to 19. The initial dynamical system above was transformed to a projected dynamical system to the space of $T$, $\mathcal{E}$ and $M = \sum_n X_{n}/N$, as also done in Ref.~\cite{Gelbrecht2021}. In this projected space the box and tessellation we used was from -2 to 10 for $M$, 0 to 50 for $\mathcal{E}$ and 230 to 350 for $T$ with 101 points in each dimension.



\subsection*{Cell genotypes}
The cell differentiation model MultiFate proposed in \cite{zhu2022synthetic} is given by 
\begin{align*}
    \dot{A_i} = \alpha + \beta \frac{ B_i^n }{ 1 + B_i^n } - A_i, \quad i = 1, 2, 3, 
\end{align*}

with, for each $i$,

\begin{align*}
    B_i = \frac{2A_i^2}{ K_d + 4(A_1 + A_2 + A_3) + \sqrt{ K_d^2 + 8(A_1 + A_2 + A_3) K_d } }, 
\end{align*}

with parameters $\alpha = 0.8$, $\beta = 20$, $K_d = 1$, $n = 1.5$. 
The state space grid ranged the interval $[0, 100]$ for all $3$ dimensions, with $100$ grid cells per dimensions.

\subsection*{Turbulent flow}

We reproduce here the equations resulting of a Galerkin projection of a the stream function of a fluid limited to a finite cell volume, describing a low dimensional turbulent shear flow model, by~\cite{moehlis2004low}:

\begin{align*}
\frac{da_1}{dt} &= \frac{\beta^2}{Re} - \frac{\beta^2}{Re} a_1 - \sqrt{\frac{3}{2}} \frac{\beta \gamma}{\kappa_{\alpha\beta\gamma}}a_6 a_8 + \sqrt{\frac{3}{2}}\frac{\beta \gamma}{\kappa_{\beta\gamma}}a_2 a_3\\
\frac{da_2}{dt} &= -\left( \frac{4\beta^2}{3} + \gamma^2\right) \frac{a_2}{Re} + \frac{5\sqrt{2} \gamma^2}{3\sqrt{3} \kappa_{\alpha\gamma}}a_4 a_6 - \frac{\gamma^2}{\sqrt{6} \kappa_{\alpha \gamma}} a_5 a_7 - \frac{\alpha \beta \gamma}{\sqrt{6}\kappa_{\alpha \gamma} \kappa_{\alpha\beta\gamma}}a_5 a_8 - \sqrt{\frac{3}{2}}\frac{\beta\gamma}{\kappa_{\beta\gamma}}a_1 a_3 - \sqrt{\frac{3}{2}}\frac{\beta\gamma}{\kappa_{\beta\gamma}}a_3 a_9\\
\frac{da_3}{dt} &= -\frac{\beta^2 +\gamma^2}{Re} a_3 + \frac{2}{\sqrt{6}} \frac{\alpha \beta \gamma}{\kappa_{\alpha\gamma}\kappa_{\beta\gamma}}(a_4 a_7 + a_5 a_6) + \frac{\beta^2(3\alpha^2 +\gamma^2) - 3\gamma^2(\alpha^2 +\gamma^2)}{\sqrt{6} \kappa_{\alpha\gamma}\kappa_{\beta\gamma}\kappa_{\alpha\beta\gamma}} a_4 a_8\\
\frac{da_4}{dt} &= - \frac{3\alpha^2 + 4\beta^2}{3Re} a_4 - \frac{\alpha}{\sqrt{6}} a_1 a_5 -\frac{10\alpha^2}{3\sqrt{6}\kappa_{\alpha\gamma}}a_2 a_6 -\sqrt{\frac{3}{2}}\frac{\alpha\beta\gamma}{\kappa_{\alpha\gamma}\kappa_{\beta\gamma}}a_3 a_7 -\sqrt{\frac{3}{2}}\frac{\alpha^2\beta^2}{\kappa_{\alpha\gamma}\kappa_{\beta\gamma}\kappa_{\alpha\beta\gamma}}a_3 a_8 -\frac{\alpha}{\sqrt{6}} a_5 a_6\\
\frac{da_5}{dt} &= -  \frac{\alpha^2 + \beta^2}{Re} a_5 + \frac{\alpha}{\sqrt{6}} a_1 a_4 + \frac{\alpha^2}{\sqrt{6}\kappa_{\alpha\gamma}}a_2 a_7 -  \frac{\alpha\beta\gamma}{\sqrt{6}\kappa_{\alpha\gamma}\kappa_{\alpha\beta\gamma}}a_2 a_8 + \frac{\alpha}{\sqrt{6}} a_4 a_9 + \frac{2\alpha\beta\gamma}{\sqrt{6}\kappa_{\alpha\gamma}\kappa_{\beta\gamma}}a_3 a_6\\
\frac{da_6}{dt} &= -\frac{3\alpha^2 + 4\beta^2 + 3\gamma^2}{3 Re}a_6 + \frac{\alpha}{\sqrt{6}} a_1 a_7 + \sqrt{\frac{3}{2}}\frac{\beta\gamma}{\kappa_{\alpha\beta\gamma}}a_1 a_8 + \frac{10(\alpha^2 - \gamma^2)}{3\sqrt{6} \kappa_{\alpha\gamma}} a_2 a_4 - 2 \sqrt{\frac{2}{3}} \frac{\alpha\beta\gamma}{\kappa_{\alpha\gamma}\kappa_{\beta\gamma}}a_3 a_5 + \frac{\alpha}{\sqrt{6}}a_7 a_9 + \sqrt{\frac{3}{2}} \frac{\beta\gamma}{\kappa_{\alpha\beta\gamma}}a_8 a_9\\
\frac{da_7}{dt} &= -\frac{\alpha^2 + \beta^2 + \gamma^2}{Re} a_7 - \frac{\alpha}{\sqrt{6}}(a_1 a_6 + a_6 a_9) + \frac{\gamma^2 - \alpha^2}{\sqrt{6}\kappa_{\alpha\gamma}}a_2 a_5 + \frac{\alpha\beta\gamma}{\sqrt{6}\kappa_{\alpha\gamma}\kappa_{\beta\gamma}}a_3 a_4\\
\frac{da_8}{dt} &= -\frac{\alpha^2 + \beta^2 + \gamma^2}{Re} a_8 + \frac{2\alpha\beta\gamma}{\sqrt{6}\kappa_{\alpha\gamma}\kappa_{\alpha\beta\gamma}} a_2 a_5  + \frac{\gamma^2(3\alpha^2 - \beta^2 + 3\gamma^2)}{\sqrt{6} \kappa_{\alpha\gamma}\kappa_{\beta\gamma}\kappa_{\alpha\beta\gamma}}a_3 a_4\\
\frac{da_9}{dt} &= -\frac{9\beta^2}{Re} a_9 + \sqrt{\frac{3}{2}}\frac{\beta \gamma}{\kappa_{\beta\gamma}}a_2 a_3 - \sqrt{\frac{3}{2}}\frac{\beta\gamma}{\kappa_{\alpha\beta\gamma}}a_6 a_8\\
\kappa_{\alpha\gamma} &= \sqrt{\alpha^2 + \gamma^2}\\
\kappa_{\beta\gamma} &= \sqrt{\beta^2 + \gamma^2}\\
\kappa_{\alpha\beta\gamma} &= \sqrt{\alpha^2+ \beta^2 + \gamma^2}
\end{align*}

Global parameters are    $L_x = 1.75\pi$, $L_z = 1.2\pi$, $\alpha = 2\pi/L_x$; $\beta = \pi/2$,  $\gamma = 2\pi/L_z$ 



\subsection*{Ecosystem dynamics}
The population dynamics model of competing species is described by the following equations \cite{huisman2001fundamental}, for $n$ species and $3$ resources:
\begin{align*}
    \dot{N_i} &= N_i [\mu_i(R_1, R_2, R_3) - m ], \quad i = 1, \cdots, n, \\
    \dot{R_j} &= D (S - R_j) - \sum_{i=1}^n c_{ji} \mu_i(R_1, R_2, R_3) N_i, \quad j = 1, 2, 3.
\end{align*}

The term $\mu_i(R_1, R_2, R_3)$ is given by, for each $i = 1, \cdots, n$,
\begin{align*}
    \mu_i(R_1, R_2, R_3) = \min\left( \frac{r R_1}{K_{1i} + R_1}, \frac{r R_2}{K_{2i} + R_2}, \frac{r R_3}{K_{3i} + R_3} \right).
\end{align*}

The parameters used in the figure are: $n = 5$, $m = 0.25$, $S = 10$, $r = 1.0$, 
\begin{align*}
        K &= \begin{bmatrix}
        0.20 & 0.05 & 1.00 & 0.05 & 1.20 \\
        0.25 & 0.10 & 0.05 & 1.00 & 0.40 \\ 
        0.15 & 0.95 & 0.35 & 0.10 & 0.05 \\
        \end{bmatrix}, \\
        c &= \begin{bmatrix}
        0.20 &0.10 &0.10 &0.10 &0.10\\
        0.10 &0.20 &0.10 &0.10 &0.20\\
        0.10 &0.10 &0.20 &0.20 &0.10 \\
    \end{bmatrix},
\end{align*}

and $D$ is given in the axis of the figure. The grid in state space was defined in the interval $[0, 60]$ for all dimensions, and discretized to include $300$ grid squares. The initial conditions in a grid in the same interval, but with only $2$ grid squares to dimension, rendering $2^8$ initial conditions for the $8$-dimensional state space. 


\subsection*{Hénon map}
A simple yet famous 2-dimensional discrete time map with constant Jacobian due to H\'enon
\begin{align*}
    x_{n+1} &= 1 - ax^2_n+y_n \\\\
    y_{n+1} & = bx_n
\end{align*}
with $b = 0.3$ and $a$ ranging from 1.2 to 1.25. We used the state space box from -2.5 to 2.5 with 500 cell points in each dimension.

\subsection*{Second order Kuramoto oscillators on networks}

This model of coupled oscillators reproduces the dynamics of synchronous generators coupled over a power grid. The equations of the nodes are: 
\begin{align*}
\dot \phi_n &=  \omega_n\\
\dot \omega_n &= \pm 1 - 0.1\omega - K \sum_j A_{ij}sin(\Phi_i - \Phi_j),
\end{align*}
where $\phi_n$ and $\omega_n$ are the phase and the frequency of the oscillator $n$. The adjacency matrix $A_{ij}$ contains all the information about the coupling of the system and is taken from a random regular graph of degree 3. The leading coefficient of the second equation is 1 when $n$ is odd and -1 otherwise. K is the coupling coefficient between oscillators that is used as a parameter for the study of the basins fractions. The panel (d) Figure 3 of the article has been processed such that basins with less than 4\% of the basins fractions are aggregated into the cluster called ``outliers''. 

\subsection*{Kuramoto coupled oscillators on networks}
This is the classical network of $N$ phase oscillator with a global coupling: 
\begin{align*}
\dot \phi_n &= \omega_n - \frac{K}{N} \sum_j sin(\phi_i - \phi_j),
\end{align*}
Frequencies of the individual oscillators $\omega_n$ are spread evenly within the interval $[-1, 1]$. 

\section{Computational performance comparison}
The benchmarks presented in Fig. \ref{fig:benchmarks} provide a comparison between techniques for finding attractors for a discrete and continuous dynamical system.
\begin{figure}[!h]
    \centering
    \includegraphics[width = 14cm]{figures/mappers}
    \caption{Benchmark comparison between all methods for finding attractors and their basins in DynamicalSystems.jl. Note that the featurizing method scales quadratically with the number of initial conditions, because the DBSCAN algorithm scales quadratically.}
    \label{fig:benchmarks}
\end{figure}

\section{Comparison with GAIO and cell mapping techniques} 

The numerical approximations of the attractors and their basins of attraction can be achieved with other well known numerical tools. The cell mapping and the GAIO algorithms rely both on a subdivision of the state space. In its simplest form \cite{sun2018cell}, the cell mapping technique transforms the dynamical system into a discrete mapping. Each cell of the state space is mapped to another cell following the dynamics during a fixed time $T$. The new representation of the dynamical system is a directed graph where each node represents an initial condition on the tessellated phase space and a single edge starts from this node to another one in the graph. Once the full mapping has been obtained, efficient search algorithms inspired from graph theory approximate the basins and the attractors. The computational effort is centered around the construction of the mapping and depends directly on the discretization of the state space. The computational complexity explodes with the system dimension and hence limits its practical use to low dimensional systems.

The Global Analysis of Invariant Objects (GAIO) algorithm \cite{dellnitz2001algorithms} is also based on the discretization on a region but only a subset of cells is subdivided following the dynamics of the system. An iterative process allows to approximate accurately the attractor manifold and also other invariant sets embedded in the state space. The complexity only depends on the dimension of the manifold not on the dimension of the state space, which is a considerable gain over the cell mapping. Since the method is focused on global attracting sets, it is not usable for multistable systems, which are the main focus of our work. 

These two techniques are hard to apply in estimating the basins fractions across a parameter value. The cell mapping technique requires a full description of the state space for each parameter, making the random sampling of the phase space impossible. The GAIO technique can continue a global attractor but will fail if two or more exist at any parameter value~\cite{gerlach2020set}.

We should mention another algorithm using interval arithmetic for continuation of the dynamics \cite{arai2009database}. The continuation is performed in a database space using Conley indices and Morse decomposition. Although very effective for low-dimensional maps, it lacks the flexibility and ease of use of our proposed method. 



\bibliography{REFERENCES}

\end{document}
