\chapter{Introduction}

\section[networks]{Networks}

Several natural and artificial systems are composed of separate entities that interact together, forming networks - or, at least, they can be approximately modelled as networks. Often, these interactions generate complex behaviors, which would not exist without the interactions. For instance, neural circuits blabla.

A major area of research today is to understand precisely how this large-scale complex dynamics emerges from the interactions between units in networks. This ranges from setting up experiments XX, modelling to a high precision, and also building the basic theory that aims to describe the fundamental aspects of these networks. For this thesis we have focused on the latter case, aiming to study the fundamental behavior of simple, non-specific, networks. To do this, we have relied on a second layer of abstraction: often, networks can be modelled as dynamical systems following ODEs of the general form: 
%
\begin{align}
    \dot{x}_i = f(x_i) + g(x)
\end{align}

where XX. Introduce topology, connections. Examples? power grids and neural networks and kuramoto?

This abstraction is quite helpful, because systems of this form can be studied using techniques from dynamical systems theory. XX?

Among the plethora of important dynamics arising in networks, we have in this thesis focused on three particular behaviors: malleability, multistability and metastability. Although separate, they are intrisically related, as we will see. For the first two, we have focused on how they are controlled by the network's topology.

% \section[malleability]{Malleability}

The first behavior we have studied is \textit{dynamical malleability}, which refers to the capacity of a network to change its dynamics when the individual parameters of units or connections are changed. We studied this behavior in Kuramoto oscillator networks of the form 

\begin{align}
    \dot{\theta} ...
\end{align}

These networks serve as paradigmatic models to understand emergent behavior - in particular, synchronization - in complex networks. They can be derived as an approximation for generic coupled limit cycle oscillators under weak coupling \cite{}. Although it does not model a particular real-world system, it has been used as a simple model for large-scale brain networks \cite{} and power grids \cite{} ?. Originally used to describe chemical oscillations XX.




\section[multistability]{Multistability}


\section[metastability]{Metastability}


% Moving now beyond the concrete research that we have done, and into a bit of speculation and plans for the future...
usefulness for computations?