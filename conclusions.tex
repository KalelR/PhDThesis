\chapter{Conclusions}\label{chap:conclusions}

Science is typically reductionist \cite{strogatzsyncbook}. We break a hard problem into smaller parts that are easier to understand separately. We have achieved tremendous success with this effort. But we have not solved everything; indeed, we have found out that putting everything back together can be quite complicated: interactions between even simple units can generate complicated behavior that is not present in any one of the units by themselves.
The field of complex systems arose from the need to understand emergent phenomena - to (re)construct the full system's behavior from knowledge of its parts. A major challenge still today is to develop tools that allows us to characterize and figure out this complicated emergent behavior.

One such complicated behavior is the coexistence of multiple stable solutions to the same equations with the same parameters - \textit{multistability}! How do these solutions come about, where they are situated, how they are they separated in state space - these are all questions under active research \cite{feudel2008complex, pisarchik2022multistability, zhang2021basins}. 

Some of these stable solutions may correspond to synchronized regimes, which brings into light another important phenomenon: \textit{synchronization}. Here again the field of complex systems has to contend with another problem: how individually distinct units can cooperate together and start to operate in unison, in a beautiful example of an emergent phenomenon. The study of synchronization - both frequency and phase synchronization - also has important practical motivations, for instance in the study of power grids. In power grids, and other complex networks, understanding the robustness of solutions, in particular of synchronized solutions has been an object of active research. 

% ------------------------------- malleability ------------------------------- %
Combining these two research areas, Chapter \ref{chap:malleability} investigated the robustness of solutions in a complex network of Kuramoto oscillators, a paradigmatic model for studies on synchronization phenomena and complex networks in general. The idea was to investigate how the network behaves - how the solutions change - when we alter the parameter of a single unit in the network. We found that the \textit{dynamical malleability} of the network depends on how strongly coupled the units are, and the topology of the connections. 
Roughly, we showed that for very weak coupling strength the individual tendencies of the oscillators win and most of them oscillate incoherently. For sufficiently strong coupling, most of the oscillators become phase locked - they oscillate at the same frequency. This is the same behavior as in all-to-all networks (see Sec.\ref{method:sec:kuramoto}). The spatial pattern of the phases, which we can measure via the degree of phase synchronization, was then determined by the topology. Networks dominated by short-range connections tended to have short-range patterns (phase desynchronized), while networks dominated by long-range connections tended to have long-range patterns (phase synchronized). These networks typically have multiple attractors coexisting, but most of the attractors, including the most synchronized attractor, follow this tendency.
In parameter space, phase synchronization in these networks lives in the region of sufficiently high coupling strength and number of long-range connections. Changing the parameters toward this region therefore makes the system undergo a transition to phase synchronization. We showed that precisely during this transition their dynamical malleability increases considerably. To the point that changing a single unit radically alters the pattern of phases in the network, potentially changing it from phase synchronized to phase desynchronized. 

The mechanism for this dynamical malleability is two-fold. First, it is related to increased sample-to-sample fluctuations near a phase transition \cite{hong2007entrainment, hong2007finitesizescalingpre}. This mechanism does not require multistability. In fact, suppose the systems have a single attractor, like the randomly connected networks. Each change to a parameter of a unit leads to a different dynamical system, which may have a different attractor. In particular, the transition to phase synchronization of this attractor may occur at different coupling strength values, earlier or later compared to the system before the change. If we enact this change but keep the coupling strength fixed, we switch to an attractor that has a smaller or larger value of phase synchronization - this is the fluctuation from one sample to another. If the systems have multiple attractors, this effect is still there, but there is the added possibility of switching to other attractors, which might be even more different. The multistability increases the possible fluctuations that may occur. This explains our observation that for Watts-Strogatz networks the malleability and multistability seem to go hand in hand. It also explains why these networks have a considerably larger malleability than the distance-dependent networks, which do not seem to be multistable.

An important new concept in the area of complex systems is that of global stability, typically taken to mean basically the relative size of the basin of attraction of each attractor - attractors that occupy larger regions of state space are more globally stable, in this view \cite{menck2013how}. Considering a trajectory on an attractor, bigger basins of attraction mean that bigger perturbations are on average needed in order to kick the trajectory across the basin boundary and into another attractor. This is, of course, a simplification \cite{krakovska2023resilience}, but it highlights the importance given to studying perturbations applied to the state a system. And, in general, more attractors means they are sharing state space more and therefore the global stability is smaller, meaning the system is less robust (or less resilient, depending on terminology \cite{krakovska2023resilience}). In this work we show that multistability affets the robustness of the system in another way: by affecting its malleability. So not only is it dangerous to kick the state of the system, it is also dangerous to change its parameters - even the parameter of one single unit!

Another important observation was the study of how malleability, and multistability, depend on the topology of the system. Topologies that put the systems in the vinicity of a transition to phase synchronization, which were in the small-world range, made it very malleable. An important question that is left for future work is why these specific topologies lead to a higher number of attractors - which properties do they possess that lead to the emergence of the attractors, compared to, say, the random topologies, which do not induce multistability? The distance-dependent networks also do not seem to be multistable, a factor that would also be interesting to investigate. 

A related question is about the generality of these results. Malleability due to STS fluctuations seems to be quite general, being extensibly described in statistical physics literature \cite{sornette2006critical}. We also described it initially in a network of spiking neurons \cite{budzinski2020synchronization}, and observed it in the Kuramoto model under different topologies of distributions of the natural frequency, and under other models, such as a simple model of excitable cells. We are confident the multistability results will somehow also generalize - supported by the available evidence from other works - but this is also object of future research. Understanding better the mechanisms generating this multistability will also help answer this.

% ------------------------------ excitable units ----------------------------- %
In a similar vein, we also investigated how multistability comes about when excitable neurons are coupled diffusively. 
powerful mechanism for generating multiple attractors.
studied mainly coupling in x and y, but just in x is already enough. 

based on the mechanism and preliminary results, we conjecture that topology plays a key role in dictating which attrators emerge. this is similar to kuramoto, and a more in depth comparison is definitely warranted.

didnt focus on sync, but we couldve: for I=3.1 XX.
came into this problem when trying to figure out THE LOMBA. found out chaotic saddle was important but also slow region of LC generating multistability. Now we understand how slowness can help generate attractors in a simpler system, we can also go back to the original problem. 
generality of results is still a question.

% ------------------------------- Attractors.jl ------------------------------ %
XX


% ------------------------------- metastability ------------------------------ %
everything is a matter of time scales. if your observation time is long enough, compared to the relaxation time to the attractors, then you see attractors. if it is not long enough, you dont have time to reach the attractors; you thus see transients. the former is a common case, but so is the latter, for instance in the brain where a lot of stuff is going on fast FAST I SAY. this is made more complicated due to the fact there are many mechanisms that can generate long transients. we saw one example of this in chap 3 with the excitability 

to illustrate the importance of transients allow us to do a quick detour into TURING MACHINES. computation is done on the transient. 
plethora of observations in the literature. important works, lots of definitions, some proposals of mechanisms. lack of conceptual framework. 


% Besides allowing for a clearer understanding of works, it can also help in future studies looking into the mechanisms behind observations and into why metastability is so ubiquitous - is there any advantage of using long transients for doing computations, as opposed to using attractors \cite{koch2024XX}?  
despite the initial motivation being to provide a nice definition, the work outgrew this and became a conceptual framework to think about metatsability, in what we believe is an important step toward a general understanding of transient dynamics. 

computations, advantages

% ------------------------------------ END ----------------------------------- %
complex networks, topology, sync (hearken back to daad proposal)
toward complete understanding of the behavior of the system, on the attractors, and before 
trying to put all those pieces together

