\chapter{Conclusions}\label{chap:conclusions}

%Along the way I also collaborated to provide open-source code to the community. And also have left several possible paths to be explored in the future, the most salient of which are:
%
\begin{itemize}
    \item Malleability: 
        \subitem how multistability arises in Kuramoto networks with heterogeneous frequencies.
        \subitem how attractors in multistable systems can be buried by adding noise.
    \item Multistability in excitable systems
        \subitem what is the effect of changing the topology in the excitable networks, which attractors may or may not arise depending on these changes.   
        \subitem using the knowledge gained from the coupled excitable systems, go back to the bursting neuron attractors and elucidate in detail the mechanism generating multistability there.
    \item Metastability
        \subitem what are the advantages of using metastability to perform computations? What are the mechanisms underlying observations of metastability in the literature?   
\end{itemize}


Besides allowing for a clearer understanding of works, it can also help in future studies looking into the mechanisms behind observations and into why metastability is so ubiquitous - is there any advantage of using long transients for doing computations, as opposed to using attractors \cite{koch2024XX}?  
We believe this conceptual framework is an important step toward a general understanding of transient dynamics. 