\section*{Abstract}
\addcontentsline{toc}{section}{Abstract}  % Add to TOC

Many systems in nature and in theory exhibit emergent behavior, where relatively simple parts interact to create a complex global behavior that is not present in any of the parts alone. Many dynamical systems with emergent behavior can be modeled as networks, in which individual units interact with each other along specified connections. An important phenomenon that can be emergent is multistability, the coexistence of many stable solutions - attractors - in a dynamical system with fixed parameters. Multistability is observed, for instance, in power grids, brain circuits, and ecological networks. This has important consequences: a multistable system operating on a particularly desirable attractor may not be safe, as a perturbation in the state of the system can cause it to switch to another coexisting attractor. On the other hand, coexistence of attractors may be useful for systems performing computations such as memory. Multistability can arise from the interactions of the multiple subunits, but the specific mechanisms that generate it are not fully known. It can also coexist with another emergent phenomenon in networked systems: synchronization, in which the interactions between units cause them to adjust their rhythms toward a collective motion. For instance, frequency synchronization occurs when units with different natural frequencies lock their oscillations onto a common frequency. The phases of their oscillations may also cluster together, in a phenomenon called phase synchronization.  Synchronized attractors can coexist with each other and with unsynchronized attractors. In this case, understanding the robustness of the attractors becomes relevant - for instance, the attractor with frequency synchronization is required for proper operation of power grids, and switching to an undesired attractor may correspond to a blackout. 

After introducing the fundamental theoretical concepts used in this thesis (Chapter 2), we move to the first work in the thesis (Chapter 3), which studies networks of Kuramoto oscillators with heterogeneous frequencies, a paradigmatic model for studies on synchronization and dynamics of complex networks. By increasing the strength of the inter-unit coupling and by adjusting the topology of connections in the network, these systems display a transition toward phase synchronization. Furthermore, near this transition the networks become highly sensitive to changes in parameters of individual components, such that even changes to single units can alter the dynamics of the entire network. We say that the networks attain a high dynamical malleability and show that this increase in the malleability is due to two effects: increase in sample-to-sample fluctuations near a phase transition and multistability. This work therefore contributes to our understanding of robustness of complex networks, in particular how their malleability and multistability depend on their topology. 

In the second work of this thesis (Chapter 4), we focus deeper on mechanisms for multistability, and investigate a network of diffusively coupled excitable neurons. Separately, a unit has only one attractor, a stable equilibrium. Before reaching this attractor, however, some trajectories in the unit's state space must go through long excursions (excitations) along an excitability region. Although the units separately do not have oscillations, we show that a rich variety of stable oscillations can emerge and coexist in the coupled networks. Two coupled units can already have multiple coexisting attractors, with periodic or quasiperiodic oscillations. Going to ten coupled units many more attractors can emerge, including a chaotic attractor. We uncover the bifurcations giving rise to these attractors, and explain the qualitative mechanism behind them. We show that the coupling between the units interacts with the excitability region of their state space and manages to repeatedly reinject them there, where they stay effectively trapped. This serves as a simple yet powerful mechanism for the creation of multistability in networks, and provides insights into how the topology of networks affects their multistability. 

Interestingly, the attractors in the previous case arise due to the interaction with the transient dynamics of the units, in the excitability region. Transient dynamics can also play important roles more broadly. In particular, long-lived transients are a ubiquitous behavior in neural activity. In this context, the third work in this thesis (Chapter 5) provides a general conceptual framework for long-lived transients. Looking at the neuroscience literature, we argue that long-lived transients are the key concept behind metastability, a term that is often used without a clear definition. We use the concept of almost-invariant regions - sets in state space wherein trajectories stay for a long time before leaving - and argue that metastable regimes in time correspond to trajectories visiting an almost-invariant region in state space. With this, we identify general dynamical properties of metastability. Then, we discuss many mechanisms that can generate metastability, and provide a classification of subtypes of metastability, which neatly includes previous works in the literature. Our hope is that this framework aids future research in neuroscience, and even other areas in which metastability occurs, such as climate science.

Finally, this thesis also describes a work (Chapter 2.1.7) developing and implementing state-of-the-art algorithms for finding attractors and their basins of attraction, including the possibility to do so in a continuation scenario over a parameter range. These algorithms were used throughout the thesis, and are available in an efficient open-source package for studying dynamical systems. 






% All work and no play makes Jack a dull boy.
% All work and no play makes Jack a dull boy.
% All work and no play makes Jack a dull boy.
% All work and no play makes Jack a dull boy.
% All work and no play makes Jack a dull boy.
% All work and no play makes Jack a dull boy.
% All work and no play makes Jack a dull boy.
% All work and no play makes Jack a dull boy.
% All work and no play makes Jack a dull boy.
% All work and no play makes Jack a dull boy.
% All work and no play makes Jack a dull boy.
% All work and no play makes Jack a dull boy.
% All work and no play makes Jack a dull boy.
% All work and no play makes Jack a dull boy.
% All work and no play makes Jack a dull boy.
% All work and no play makes Jack a dull boy.
% All work and no play makes Jack a dull boy.
% All work and no play makes Jack a dull boy.
% All work and no play makes Jack a dull boy.
% All work and no play makes Jack a dull boy.
% All work and no play makes Jack a dull boy.
% All work and no play makes Jack a dull boy.
% All work and no play makes Jack a dull boy.
% All work and no play makes Jack a dull boy.
% All work and no play makes Jack a dull boy.
% All work and no play makes Jack a dull boy.
% All work and no play makes Jack a dull boy.
% All work and no play makes Jack a dull boy.
% All work and no play makes Jack a dull boy.