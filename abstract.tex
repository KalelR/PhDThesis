\section*{Abstract}
\addcontentsline{toc}{section}{Abstract}  % Add to TOC

Field of complex systems, emergent phenomena. One such is multistability. Another is sync. Path towards the attractors is also important - transients. These are the objects of study in this thesis, which is subdivided into three main works. In the first, we study the robustness of solutions of phase oscillator networks. Malleability. 2 main factors: sts and multistabiltiy. Also study the emergence of multistability in coulped excitable neurons. We show a rich coexistence of oscilations arising from excitability, with only stable equilibrium. With two units XX, with more XX. We describe the different bifurcations giving rise to the attractors here and also provide a qualitative mechanism that describes all the attractors and also generalizes to more units. Then, switching the focus to transients, in particular long transients, metastability.

Many systems in nature and in theory display emergent behavior, in which relatively simple subunits interact together to create a complicated global behavior which is not present in any of the units alone. An important example of this is multistability, the coexistence of many stable solutions - attractors - to a dynamical system with fixed parameters. 


% All work and no play makes Jack a dull boy.