\section*{Abstract}
\addcontentsline{toc}{section}{Abstract}  % Add to TOC

Many systems in nature and in theory display emergent behavior, in which relatively simple subunits interact together to create a complicated global behavior that is not present in any of the units alone. An important phenomenon that can be emergent is multistability, the coexistence of many stable solutions - attractors - in a dynamical system with fixed parameters. Multistability is observed for instance in power grids, brain circuits and ecological networks. It has important consequences: a multistable system operating on a particularly desirable attractor may not be safe, as a perturbation in the state of the system can cause it to switch to another coexisting attractor. On the other hand, coexistence of attractors may be useful for systems performing computations such as memory. In networked systems, multistability can arise from the interactions of the multiple subunits, but the specific mechanisms that generate it are not fully known. Another emergent phenomenon in networked systems is synchronization, in which the interactions between units cause them to adjust their rhythms toward a collective motion. For instance, frequency synchronization occurs when units with different natural frequencies lock their oscillations to a common frequency. It may also happen that the phases of their oscillations cluster together, in a phenomenon called phase synchronziation. 

Synchronization can coexist with multistability, as synchronized attractors can coexist with each other and with unsynchronized attractors. In this case, understanding the robustness of the attractors can be of relevance - for instance, the attractor with frequency synchronization is required for proper operation of power grids; switching to an undesired attractor may correspond to a blackout. The first work in this thesis studies networks of Kuramoto oscillators with heterogenous frequencies, a paradigmatic model for studies on synchronization and dynamics of complex networks. We show that, by increasing the strength of the inter-unit coupling and by adjusting the topology of connections in the network, these systems display a transition toward phase synchronization. We then show that a near this transition the networks become very sensitive to changes in parameters of single units - we say the networks have a high dynamical malleability. Changes to single units can alter the whole network's dynamics. We show that this increase is due to two effects: increase in sample-to-sample fluctuations near a phase transition and multistability. This work therefore contributes to our understanding of robustness of complex networks, in particular how their malleability and multistability depend on their topology. 

In the second work of this thesis, we focus deeper on mechanisms for multistability, and investigate a network of diffusively coupled excitable neurons. Individually, a unit has only one attractor, a stable equilibrium. Trajectories in a region of their state space must go through long excursions before reaching the attractor. We show that a rich variety of stable oscillations can emerge and coexist in the coupled networks, even though the units separately do not have oscillations. We uncover the bifurcations giving rise to these oscillating attractors, and explain the qualitative mechanism behind them. We show that the coupling between the units interacts with the excitability region of their state space and manages to repeatedly reinject them there, effectively trapping the units in the excitability region. This is a powerful mechanism for the creation of multistability in networks. 

Interestingly, the attractors arise here due to the interaction with the transient dynamics of the units, in the excitability region. Transient dynamics can play important roles more broadly. In particular, long transients are an ubiquitous behavior seen in neural activity, typically refered to as metastable regimes. The third paper in this thesis studies this behavior in the context of neural systems. We provide a general conceptual framework for metastability, looking at its dynamical properties and mechanisms that can generate it.

Finally, in this thesis we also present a work developing and implementing algorithms for finding attractors and their basins of attraction, including the possibility to do so in a continuation scenario over a parameter range. 






% All work and no play makes Jack a dull boy.
% All work and no play makes Jack a dull boy.
% All work and no play makes Jack a dull boy.
% All work and no play makes Jack a dull boy.
% All work and no play makes Jack a dull boy.
% All work and no play makes Jack a dull boy.
% All work and no play makes Jack a dull boy.
% All work and no play makes Jack a dull boy.
% All work and no play makes Jack a dull boy.
% All work and no play makes Jack a dull boy.
% All work and no play makes Jack a dull boy.
% All work and no play makes Jack a dull boy.
% All work and no play makes Jack a dull boy.
% All work and no play makes Jack a dull boy.
% All work and no play makes Jack a dull boy.
% All work and no play makes Jack a dull boy.
% All work and no play makes Jack a dull boy.
% All work and no play makes Jack a dull boy.
% All work and no play makes Jack a dull boy.
% All work and no play makes Jack a dull boy.
% All work and no play makes Jack a dull boy.
% All work and no play makes Jack a dull boy.
% All work and no play makes Jack a dull boy.
% All work and no play makes Jack a dull boy.
% All work and no play makes Jack a dull boy.
% All work and no play makes Jack a dull boy.
% All work and no play makes Jack a dull boy.
% All work and no play makes Jack a dull boy.
% All work and no play makes Jack a dull boy.