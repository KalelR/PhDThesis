\section*{Zusammenfassung}
\addcontentsline{toc}{section}{Zusammenfassung}  % Add to TOC

Viele Systeme in der Natur und in der Theorie weisen ein emergentes Verhalten auf, bei dem relativ einfache Teile interagieren, um ein komplexes Gesamtverhalten zu erzeugen, das in keinem der Teile allein vorhanden ist. Viele dynamische Systeme mit emergentem Verhalten können als Netzwerke modelliert werden, in denen einzelne Einheiten entlang bestimmter Verbindungen miteinander interagieren. Ein wichtiges Phänomen, das emergent sein kann, ist die Multistabilität, die Koexistenz vieler stabiler Lösungen - Attraktoren - in einem dynamischen System mit festen Parametern. Multistabilität wird zum Beispiel in Stromnetzen, Gehirnschaltungen und ökologischen Netzwerken beobachtet. Dies hat wichtige Konsequenzen: Ein multistabiles System, das auf einem besonders wünschenswerten Attraktor arbeitet, ist möglicherweise nicht sicher, da eine Störung des Systemzustands dazu führen kann, dass es zu einem anderen koexistierenden Attraktor übergeht. Andererseits kann die Koexistenz von Attraktoren für Systeme, die Berechnungen durchführen, z. B. für den Speicher, nützlich sein. Multistabilität kann durch die Interaktion mehrerer Untereinheiten entstehen, aber die spezifischen Mechanismen, die sie erzeugen, sind nicht vollständig bekannt. Sie kann auch mit einem anderen emergenten Phänomen in vernetzten Systemen koexistieren: der Synchronisation, bei der die Interaktionen zwischen den Einheiten dazu führen, dass sie ihre Rhythmen auf eine kollektive Bewegung ausrichten. Frequenzsynchronisation tritt zum Beispiel auf, wenn Einheiten mit unterschiedlichen Eigenfrequenzen ihre Schwingungen auf eine gemeinsame Frequenz abstimmen. Auch die Phasen ihrer Schwingungen können sich aneinander angleichen, was als Phasensynchronisation bezeichnet wird.  Synchronisierte Attraktoren können miteinander und mit unsynchronisierten Attraktoren koexistieren. In diesem Fall ist es wichtig, die Robustheit der Attraktoren zu verstehen - beispielsweise ist der Attraktor mit der Frequenzsynchronisation für den ordnungsgemäßen Betrieb von Stromnetzen erforderlich, und ein Wechsel zu einem unerwünschten Attraktor kann zu einem Stromausfall führen. 

Nach einer Einführung in die grundlegenden theoretischen Konzepte, die in dieser Arbeit verwendet werden (Kapitel 2), gehen wir zur ersten Arbeit in dieser Arbeit über (Kapitel 3), die Netzwerke von Kuramoto-Oszillatoren mit heterogenen Frequenzen untersucht, ein paradigmatisches Modell für Studien über Synchronisation und Dynamik komplexer Netzwerke. Wenn man die Stärke der Kopplung zwischen den Einheiten erhöht und die Topologie der Verbindungen im Netzwerk anpasst, zeigen diese Systeme einen Übergang zur Phasensynchronisation. Darüber hinaus reagieren die Netze in der Nähe dieses Übergangs sehr empfindlich auf Änderungen der Parameter einzelner Komponenten, so dass selbst Änderungen an einzelnen Einheiten die Dynamik des gesamten Netzes verändern können. Wir sagen, dass die Netzwerke eine hohe dynamische Formbarkeit erreichen, und zeigen, dass dieser Anstieg der Formbarkeit auf zwei Effekte zurückzuführen ist: Zunahme der Fluktuationen von Probe zu Probe in der Nähe eines Phasenübergangs und Multistabilität. Diese Arbeit trägt daher zu unserem Verständnis der Robustheit komplexer Netzwerke bei, insbesondere dazu, wie ihre Formbarkeit und Multistabilität von ihrer Topologie abhängen. 

In der zweiten Arbeit dieser Dissertation (Kapitel 4) befassen wir uns eingehender mit den Mechanismen der Multistabilität und untersuchen ein Netzwerk aus diffus gekoppelten erregbaren Neuronen. Für sich genommen hat eine Einheit nur einen Attraktor, ein stabiles Gleichgewicht. Bevor sie jedoch diesen Attraktor erreicht, müssen einige Trajektorien im Zustandsraum der Einheit lange Exkursionen (Erregungen) entlang einer Erregbarkeitsregion durchlaufen. Obwohl die Einheiten für sich genommen keine Oszillationen aufweisen, zeigen wir, dass in den gekoppelten Netzwerken eine Vielzahl stabiler Oszillationen entstehen und koexistieren kann. Zwei gekoppelte Einheiten können bereits mehrere koexistierende Attraktoren mit periodischen oder quasiperiodischen Schwingungen aufweisen. Bei zehn gekoppelten Einheiten können noch viel mehr Attraktoren auftreten, einschließlich eines chaotischen Attraktors. Wir decken die Bifurkationen auf, die zu diesen Attraktoren führen, und erklären den qualitativen Mechanismus dahinter. Wir zeigen, dass die Kopplung zwischen den Einheiten mit der Erregbarkeitsregion ihres Zustandsraums interagiert und es schafft, sie immer wieder dorthin zurückzubringen, wo sie effektiv gefangen bleiben. Dies ist ein einfacher, aber wirkungsvoller Mechanismus für die Erzeugung von Multistabilität in Netzwerken und gibt Aufschluss darüber, wie die Topologie von Netzwerken deren Multistabilität beeinflusst. 

Interessanterweise entstehen die Attraktoren im vorgenannten Fall durch die Interaktion mit der transienten Dynamik der Einheiten in der Erregbarkeitsregion. Die transiente Dynamik kann auch im weiteren Sinne eine wichtige Rolle spielen. Insbesondere langlebige Transienten sind ein allgegenwärtiges Verhalten bei neuronaler Aktivität. In diesem Zusammenhang liefert die dritte Arbeit in dieser Dissertation (Kapitel 5) einen allgemeinen konzeptionellen Rahmen für langlebige Transienten. Mit Blick auf die neurowissenschaftliche Literatur argumentieren wir, dass langlebige Transienten das Schlüsselkonzept hinter der Metastabilität sind, ein Begriff, der oft ohne klare Definition verwendet wird. Wir verwenden das Konzept der nahezu unveränderlichen Regionen - Mengen im Zustandsraum, in denen sich Trajektorien lange Zeit aufhalten, bevor sie diese verlassen - und argumentieren, dass metastabile Regime in der Zeit Trajektorien entsprechen, die eine nahezu unveränderliche Region besuchen.


% Arbeiten ohne Vergnügen macht Jack zu einem langweiligen Jungen Arbeiten ohne Vergnügen macht Jack zu einem langweiligen Jungen Arbeiten ohne Vergnügen macht Jack zu einem langweiligen Jungen Arbeiten ohne Vergnügen macht Jack zu einem langweiligen Jungen Arbeiten ohne Vergnügen macht Jack zu einem langweiligen Jungen Arbeiten ohne Vergnügen macht Jack zu einem langweiligen Jungen Arbeiten ohne Vergnügen macht Jack zu einem langweiligen Jungen Arbeiten ohne Vergnügen macht Jack zu einem langweiligen Jungen Arbeiten ohne Vergnügen macht Jack zu einem langweiligen Jungen Arbeiten ohne Vergnügen macht Jack zu einem langweiligen Jungen Arbeiten ohne Vergnügen macht Jack zu einem langweiligen Jungen Arbeiten ohne Vergnügen macht Jack zu einem langweiligen Jungen Arbeiten ohne Vergnügen macht Jack zu einem langweiligen Jungen Arbeiten ohne Vergnügen macht Jack zu einem langweiligen Jungen Arbeiten ohne Vergnügen macht Jack zu einem langweiligen Jungen Arbeiten ohne Vergnügen macht Jack zu einem langweiligen Jungen Arbeiten ohne Vergnügen macht Jack zu einem langweiligen Jungen Arbeiten ohne Vergnügen macht Jack zu einem langweiligen Jungen Arbeiten ohne Vergnügen macht Jack zu einem langweiligen Jungen Arbeiten ohne Vergnügen macht Jack zu einem langweiligen Jungen Arbeiten ohne Vergnügen macht Jack zu einem langweiligen Jungen Arbeiten ohne Vergnügen macht Jack zu einem langweiligen Jungen Arbeiten ohne Vergnügen macht Jack zu einem langweiligen Jungen Arbeiten ohne Vergnügen macht Jack zu einem langweiligen Jungen Arbeiten ohne Vergnügen macht Jack zu einem langweiligen Jungen Arbeiten ohne Vergnügen macht Jack zu einem langweiligen Jungen Arbeiten ohne Vergnügen macht Jack zu einem langweiligen Jungen Arbeiten ohne Vergnügen macht Jack zu einem langweiligen Jungen Arbeiten ohne Vergnügen macht Jack zu einem langweiligen Jungen Arbeiten ohne Vergnügen macht Jack zu einem langweiligen Jungen Arbeiten ohne Vergnügen macht Jack zu einem langweiligen Jungen